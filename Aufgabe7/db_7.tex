\documentclass{article}
\usepackage{scrpage2}
\pagestyle{scrheadings}
\clearscrheadfoot

\begin{document}

\ihead{Marvin Petersen (2058712)\\♠Nico Grimm (2058712)}
\chead{21.01.2016}
\ohead{Datenbank - Praktikum}

\begin{center}
\section*{Aufgabenblatt 7}
\end{center}

\subsection*{Aufgabe 7.3}

\subsubsection*{a) Beispiel in 1.NF aber nicht 2.NF}

\begin{tabular}{|c|c|c|c|c|} \hline
StudentID & Vorname & Nachname & VorlesungID & Professor \\ \hline \hline
1 & Marvin & Petersen & 10 & Prof. Hollatz \\ \hline
1 & Marvin & Petersen & 12 & Prof. Baran \\ \hline
2 & Nico & Grimm & 10 & Prof. Hollatz \\ \hline
2 & Nico & Grimm & 11 & Prof. Jenke \\ \hline
2 & Nico & Grimm & 12 & Prof. Baran \\ \hline
3 & Max & Mustermann & 11 & Prof. Jenke \\ \hline
\end{tabular} \\[2ex]
Vorname und Nachname sind funktional abh\"angig von StudentID

\subsubsection*{b) Beispiel in 0.NF}

\begin{tabular}{|c|c|c|c|c|} \hline
StudentID & Vorname & Nachname & VorlesungID & Professor \\ \hline \hline
1 & Marvin & Petersen & 10,12 & Prof. Hollatz, Prof. Baran \\ \hline
2 & Nico & Grimm & 10,11,12 & Prof. Hollatz, Prof. Baran, Prof. Jenke \\ \hline
3 & Max & Mustermann & 11 & Prof. Jenke \\ \hline
\end{tabular} \\[2ex]
Die Attribute VorlesungID und Professor haben mehrere Attributwert.

\end{document}